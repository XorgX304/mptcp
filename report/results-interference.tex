One of the advantages of using MPTCP is the fact that multiple interfaces can be
used which makes the network faster, more reliable and less prone to congestion
(at least in theory; as multiple subflows are used to split the load). However
before we jump straight into MPTCP there are a few questions about using
multiple interfaces worth asking:

\begin{enumerate}
  \item Does it make sense to use 2 interfaces at the same time? (when compared
    to a single interface)
  \item Is there ever a penalty to using them?
  \item Is there any performance gain?
\end{enumerate}

To answer these questions several experiments where ran using different setups
(as can be seen in the setup section). Most of the results obtained in both
interfering and non-interfering experiments between 5-2.4 and 2.4-2.4 show that
the 2 interfaces always have better throughput than the single interface; this
can be seen more clearly in the CDF graphs (parallel vs single graphs - look for
appropriate graphs!!!).

% Maybe display a few graphs showing this examples of 5-2.4 and 2.4-2.4 both i
% and ni

The results obtained from these graphs also answer the second and third
questions. It is clear that these results show that there is no penalty to
having multiple interfaces (unless these interfere with each other and the
combined throughput is worse, which is not the case) because carrier sense will
make sure that the total throughput is not increased even if both interfaces are
on the same channel; the performance gain is quite significant as shown in the
results, the combined throughput for the 2 interfaces is much better than the
single. (should also mention that this would work assuming the congestion
control mechanism gives you a fair share of capacity (?))

Based on this it is safe to say that using multiple interfaces is advantageous
which also implies that MPTCP is in fact a good idea, bringing us to the next
section.

\subsubsection{Carrier Sense and cross-channel interference}
\begin{figure}[h]
 \centering
 \input{graphs/mp-int-2.4-ni.tex}
 \caption{Cross-channel interference}\label{graph:cc-interference}
\end{figure}

The cross-channel interference observed in Figure~\ref{graph:cc-interference} is
quite interesting as the throughput and utilization measurements show a
completely fair sharing between the two networks when both are active. There are
also nearly no 802.11 retransmit failures. This perfect split implies that
Carrier Sense is being employed here, effectively doing time multiplexing
between the interfaces.  This would be fine if the two networks were on the same
channel and should not transmit simultaneously, but the experiment shown in
Figure~\ref{graph:cc-interference} was performed with two networks on opposite
ends of the 2.4 Ghz spectrum (channel 1 and 11), and the two interfaces should
be able to both transmit at the same time, giving twice the throughput.

Looking at the total throughput, it is clear that some performance gain is
achieved, but clearly the gain is closer to 50\% than the 100\% one would expect
from non-interfering channels.

\begin{figure}[h]
 \centering
 \input{graphs/mp-int-2.4-i.tex}
 \caption{Same-channel interference}\label{graph:sc-interference}
\end{figure}

The results in Figure~\ref{graph:sc-interference} are perhaps even more
surprising as Carrier Sense should enforce a fairly strict time multiplexing
with two networks on the same channel, meaning the total throughput should
remain almost the same whether one or two interfaces are active. The
experimental results on the other hand show that the total throughput decreases
to almost 50\% when both interfaces are active. Clearly Carrier Sense is not
performing as it should.
