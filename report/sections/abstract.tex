This report evaluates the effect of WiFi interference on Multipath TCP. 
The results presented in this report show that there can be a performance gain 
in using Multipath TCP exclusively with WiFi for both the downlink and uplink.

Both cases show that on two idle paths Multipath TCP can attain an aggregate
throughput which is equal to the sum of both links. 
However, while the downlink contending traffic results show that the aggregate 
throughput is reduced to the capacity of the best single link, thus making its 
behaviour fair to contending flows, the uplink behaves quite differently.
Uplink results show that the aggregate throughput is generally more than that of 
the best link and only in some cases equal to the sum of both links. This 
behaviour shows that Multipath TCP on the uplink is unfair to other flows on the 
medium as it gets a larger share in total. This contradicts the second rule of 
its default congestion control algorithm; the Coupled congestion control which 
states "do no harm".
% vim:textwidth=80:colorcolumn=80:spell:
