This report evaluates the effect of WiFi interference on Multipath TCP. The 
problem considered in this report and the conclusions drawn from it have been 
based on results from several wireless experiments ran using different settings 
and layouts. The results presented show that there can be a performance gain 
in using Multipath TCP exclusively with WiFi for both the downlink and uplink.

Both cases show that on two idle paths Multipath TCP can attain an aggregate
throughput which is equal to the sum of both links. However, while the downlink 
contending traffic results show that the aggregate throughput is reduced to the 
capacity of the best single link, thus making its behaviour fair to contending 
flows, the uplink behaves quite differently.
Uplink results show that the aggregate throughput is generally more than that of 
the best link and only in some cases equal to the sum of both links. This 
behaviour shows that Multipath TCP on the uplink is unfair to other flows on the 
medium as it gets a larger share in total. This contradicts the second rule of 
its default congestion control algorithm which states "do no harm", and raises 
questions regarding how well Multipath TCP behaves with WiFi.

This report aims to evaluate how Multipath TCP compares to a single-path TCP 
connection when facing interference in a wireless environment, and how this 
might affect its behaviour.

Multipath TCP was designed for wired environments and even though some of the 
concepts and issues discussed in this report have been addressed before, there 
has been no further investigation towards how Multipath TCP might work in a 
wireless environment, hence our interest in this project. We believe that by 
further evaluating Multipath TCP in a setting and a view that has not been taken 
before, we will be able to create more opportunities for it to be improved and 
become even more widespread. This would be benefitial not only for the industry 
but also for the common user of the Internet. By widespreading the use of 
Multipath TCP in wireless environments a great number of new opportunities open 
up to what can be done with it, thus trying to better understand and improve it 
might prove to be invaluable. (Maybe this contribution bit should go to intro?)


% vim:textwidth=80:colorcolumn=80:spell:
