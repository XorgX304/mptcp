One of the goals of Multipath TCP is to use the available links to improve
reliability, and one of the ways it does so is by moving traffic off congested
paths. Considering WiFi connectivity changes drastically as a user moves around,
we felt it would be interesting to put Multipath TCP's ability to move traffic
from one link to another to the test.

\begin{figure}[h]
 \centering
 \input{graphs/mobility.tex}
 \caption{Traffic distribution with movement, downlink, non-overlapping 2.4 GHz}\label{graph:mobility}
\end{figure}

We therefore ran a test as described in \S~\ref{sec:met:setups:mobility} in
which a client moved between two APs positioned sufficiently far apart that one
link was significantly better when near any one of the APs. The results shown in
Figure~\ref{graph:mobility} tell us that Coupled is indeed doing exactly what
you would expect. More traffic goes through one interface when the user is
closer to that AP, as expected, but more importantly, it evenly distributes the
traffic when the user moves back into a position where both networks are
approximately equally strong.

We chose to only perform this test with downlink traffic since, as discussed in
\S~\ref{sec:results-fairness}, Coupled simply behaves like New Reno, making the
results uninteresting.

% vim:textwidth=80:colorcolumn=80:spell:
