One of the goals of Multipath TCP is to use the available links to improve
reliability, and it does this partly by moving traffic away from congested
paths. Considering that WiFi connectivity can change drastically as a user moves
around, we felt it would be interesting to test Multipath TCP's ability to move
traffic from one link to another.

\begin{figure}[h]
 \centering
 \input{graphs/mobility.tex}
 \caption{Traffic distribution with movement, downlink, non-overlapping 2.4 GHz}\label{graph:mobility}
\end{figure}

The results shown in Figure~\ref{graph:mobility} show that Coupled is indeed
doing exactly what we would expect. More traffic passes through one interface
when the user is closer to that AP, but more importantly, it evenly distributes
the traffic when the user moves back into a position where both networks are
approximately equally strong.

We chose to only perform this test with downlink traffic since, as discussed in
\S\ref{sec:results-fairness}, Coupled simply behaves like New Reno on the
uplink, making the these results predictable.

% vim:textwidth=80:colorcolumn=80:spell:
