Before discussing Multipath TCP with Coupled congestion control, we present
results using Multipath TCP with New Reno congestion control operating
independently for each subflow. This congestion control algorithm allows the
congestion window of each subflow to grow as if it were an independent TCP flow,
and is known to use an unfair share of the total available bandwidth\footnote{RFC 6356 - Coupled Congestion Control for Multipath Transport Protocols.}. These
results may seem unnecessary, but we believe it is useful to show
the subtle difference between running two wireless clients in parallel and one
client with multiple wireless interfaces. Using New Reno allows us to evaluate this separately from the load balancing performed by the Coupled algorithm.

% [1] Coupled Congestion Control RFC.

\subsubsection{Downlink}
\label{sec:results-mptcp-down}

\begin{figure}[h]
  \centering
  \subfloat[][2.4 GHz, same channel] {\
    \scalebox{0.70}{\input{graphs/sc-reno-down.tex}}\label{graph:sc-reno-down}
  }
  \\
  \subfloat[][2.4 GHz, non-overlapping channels] {\
    \scalebox{0.70}{\input{graphs/cc-reno-down.tex}}\label{graph:cc-reno-down}
  }
  \\
  \subfloat[][5 and 2.4 GHz] {\
    \scalebox{0.70}{\input{graphs/cb-reno-down.tex}}\label{graph:cb-reno-down}
  }
  \caption{Downlink, New Reno}\label{graph:reno-down}
\end{figure}

On the downlink, we would expect each Multipath TCP subflow to have an equal
share of the total throughput as a regular TCP flow, as there should be no
difference in interference when running both WiFi interfaces on the same machine
compared to on separate machines when each interface is only transmitting ACKs.
This can be seen in \subref{graph:sc-reno-down} through
\subref{graph:cb-reno-down} in Figure~\ref{graph:reno-down}. Again, we attribute
the discrepancies in \subref{graph:cc-reno-down} to the unstable nature of
wireless networks.

The per-subflow fairness in the downlink case is achieved in two steps. Firstly,
the wireless medium should be shared equally between the two APs (assuming they
both have data to send and their channels are equally busy) due to carrier
sense. Additionally, two flows going through an AP should get an equal share of 
its airtime. Since TCP New Reno does not try to balance the load on each flow, 
the sender would continuously try to push as much data as it can get on the link 
and this is only limited by the finite buffer at the AP. The effect of this is 
that the APs drop further incoming packets as soon as their buffers are full, 
and this notifies the sender to adapt its rate. At this point, both flows have 
the same amount of data in the APs' buffer; hence, it uses about the same time 
in transmitting each flows' data, thereby effectively sharing the airtime of the 
AP.

Note that although the flows have a fair share of the bandwidth, this does not 
mean TCP New Reno is fair as it maintains its agressinveness even after adapting 
its send rate. In reality, if the AP buffers did not overflow and hence limited 
the amount of packets in flight, TCP New Reno would continue to grow 
aggressively and cause its flows to have different amounts of access to the 
medium. This would lead a flow to take more than its fair share of the 
bandwidth, displaying then an unfair behaviour.  

\subsubsection{Uplink}
\label{sec:results-mptcp-up}

\begin{figure}[h]
  \centering
  \subfloat[][2.4 GHz, same channel] {\
    \scalebox{0.70}{\input{graphs/sc-reno-up.tex}}\label{graph:sc-reno-up}
  }
  \\
  \subfloat[][2.4 GHz, non-overlapping channels] {\
    \scalebox{0.70}{\input{graphs/cc-reno-up.tex}}\label{graph:cc-reno-up}
  }
  \\
  \subfloat[][5 and 2.4 GHz] {\
    \scalebox{0.70}{\input{graphs/cb-reno-up.tex}}\label{graph:cb-reno-up}
  }
  \caption{Uplink, New Reno}\label{graph:reno-up}
\end{figure}

Due to the unpredictable nature of wireless and varying levels of background
traffic, obtaining consistent results for these tests was a time consuming
process. The general trend we observe in Figure~\ref{graph:reno-up} is the
same seen on the downlink; each Multipath TCP subflow gets an equal
share of the total throughput to every other flow on that link. This also makes
sense conceptually; on the uplink, the 802.11 MAC provides fairness
\textbf{per 802.11 station} using carrier sense as explained in
\S\ref{sec:bg:wifi}. On the downlink fairness is provided by carrier sense
\textbf{and} the APs. Since Multipath TCP acts as one station on each network,
we would expect each subflow to get half the available capacity on each network
when using New Reno.

The equal distribution above gives Multipath TCP an aggregate throughput equal
to the sum of half of each links' capacity, and suggests that no additional interference
penalty is incurred by running both interfaces on the same client rather than on
different clients. We can also reason from this that Multipath TCP with Coupled
congestion control should theoretically be able to eventually reach the same
throughput as parallel clients if no other flows are competing on the network; 
this will be examined at in \S\ref{sec:results-performance}.

% vim:textwidth=80:colorcolumn=80:spell:
