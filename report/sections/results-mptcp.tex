Before discussing Multipath TCP with Coupled congestion control, we present
results using Multipath TCP with New Reno congestion control operating
independently for each subflow. This congestion control algorithm is known to
use an unfair share of the available capacity with Multipath TCP, as it
allows the congestion window of each subflow to grow as if it were an
independent TCP flow\footnote{RFC 6356 - Coupled Congestion Control for
Multipath Transport Protocols.}. These
results may seem unnecessary, but we believe they are useful to show
the subtle difference between running two wireless clients in parallel and one
client with multiple wireless interfaces. Using New Reno allowed us to evaluate
this separately from the load balancing performed by the Coupled algorithm.

% [1] Coupled Congestion Control RFC.

\subsubsection{Downlink}
\label{sec:results-mptcp-down}

\begin{figure}[h]
  \centering
  \subfloat[][2.4 GHz, same channel] {\
    \scalebox{0.70}{\input{graphs/sc-reno-down.tex}}\label{graph:sc-reno-down}
  }
  \\
  \subfloat[][2.4 GHz, non-overlapping channels] {\
    \scalebox{0.70}{\input{graphs/cc-reno-down.tex}}\label{graph:cc-reno-down}
  }
  \\
  \subfloat[][5 and 2.4 GHz] {\
    \scalebox{0.70}{\input{graphs/cb-reno-down.tex}}\label{graph:cb-reno-down}
  }
  \caption{Downlink, New Reno}\label{graph:reno-down}
\end{figure}

On the downlink, we would expect each Multipath TCP subflow to have an equal
share of the total throughput as a regular TCP flow, as there should be no
difference in interference when running both WiFi interfaces on the same machine
compared to on separate machines when each interface is only transmitting ACKs.
This can be seen in \subref{graph:sc-reno-down} through
\subref{graph:cb-reno-down} in Figure~\ref{graph:reno-down}. We note that in \subref{graph:cc-reno-down} the throughputs are not identical; we attribute these discrepancies to the unstable nature of wireless networks.

The per-subflow fairness in the downlink case is achieved in two ways. Firstly,
the wireless medium should be shared equally between the two APs (assuming they
both have data to send and their channels are equally busy) due to carrier
sense. Additionally, two flows going through an AP should get an equal share of 
its airtime. Since TCP New Reno does not balance the load across the subflows, each sender will continue to increase their congestion window independently, limited only by the finite buffer of the AP. The effect of this is that the APs drop further incoming packets as soon as their buffers are full, and this indicates to the senders that they should adapt their rate. At this point, both flows going through a single AP have the same amount of data in that AP's buffer. It spends the same time in transmitting each flows' data, and thus shares its airtime evenly between the flows.  

\subsubsection{Uplink}
\label{sec:results-mptcp-up}

\begin{figure}[h]
  \centering
  \subfloat[][2.4 GHz, same channel] {\
    \scalebox{0.70}{\input{graphs/sc-reno-up.tex}}\label{graph:sc-reno-up}
  }
  \\
  \subfloat[][2.4 GHz, non-overlapping channels] {\
    \scalebox{0.70}{\input{graphs/cc-reno-up.tex}}\label{graph:cc-reno-up}
  }
  \\
  \subfloat[][5 and 2.4 GHz] {\
    \scalebox{0.70}{\input{graphs/cb-reno-up.tex}}\label{graph:cb-reno-up}
  }
  \caption{Uplink, New Reno}\label{graph:reno-up}
\end{figure}

Due to the unpredictable nature of wireless and varying levels of background
traffic, obtaining consistent results for these tests was a time consuming
process. The general trend we observe in Figure~\ref{graph:reno-up} is the
same as was seen on the downlink; each Multipath TCP subflow gets an equal
share of the total throughput to every other flow on that link. This also makes
sense conceptually; on the uplink, the 802.11 MAC provides fairness
\emph{per 802.11 station} using carrier sense as explained in
\S\ref{sec:bg:wifi}. On the downlink fairness is provided by carrier sense
\emph{and} the APs. Since Multipath TCP acts as one station on each network,
we would expect each subflow to get half the available capacity on each network
when using New Reno.

This equal distribution of transmit time gives Multipath TCP an aggregate throughput equal
to the sum of half of each links' capacity, and suggests that no additional interference
penalty is incurred by running both interfaces on the same client rather than
separate clients. We can also reason from this that Multipath TCP with Coupled
congestion control should theoretically be able to eventually reach the same
throughput as parallel clients if no other flows are competing on the network; 
this will be examined at in \S\ref{sec:results-performance}.

% vim:textwidth=80:colorcolumn=80:spell:
