In this paper we have presented a number of wireless experiments showing that
multiplexing across multiple wireless interfaces on a single host may indeed
provide advantages over single interface hosts for certain network
configurations both in terms of reliability and throughput.


Further, we have shown that Multipath TCP works quite well when traffic is
primarily downlink, which is often the case on wireless networks today, as 
traffic is balanced between the available wireless networks and each host only 
uses a fair share of the total network capacity. We have also shown that 
Multipath TCP does not behave fairly with uplink traffic due to a combination of 
802.11 retransmissions and effectively unbounded buffering in the network stack.

Finally, we have discussed how the uplink problems experienced by Multipath TCP
are not unique to wireless links, and that they manifest also for regular TCP
connections as bloated buffers. Unfortunately due to time constraints, we have 
not been able to test our suggestion for a solution to this problem, and so the 
question of how to properly limit queue sizes without the presence of loss 
remains an open problem.

Due to widespread deployment and high diversity in wireless networks, there
are many potentially interesting experiments we have not been able to do due to
the limited time of the project. First and foremost, all of our experiments were 
performed in a building where several other wireless networks were present, 
making it difficult to obtain consistent results. To verify the behaviour we 
have observed, repeating the experiments in more controlled environments may be 
an appropriate next step.

Most of our experiments were also conducted with only two, static APs, and two 
or three wireless clients each with a maximum of two interfaces. Investigating 
how Multipath TCP behaves when more networks are available, or with different 
physical distributions of APs may yield significantly different results. 
As 5~GHz networks become more prevalent, testing the amount of interference 
between 5~GHz channels could also be of value. 

Additionally, we have shown that Multipath TCP experiences varying level of 
interference based on the underlying network. We see that there is no 
interference across the 2.4GHz and 5GHz bands while the interference patterns 
observed in \S\ref{sec:results-interference} suggest that some further research 
on how Carrier Sense works across channels could be warranted.

It should be noted that Multipath TCP does work sufficiently well for wired 
networks as seen in [FIND PAPER TO REFERENCE], and most of the issues observed 
in this report are in relation with its behaviour with WiFi. Problems such as 
reduced  throughput for flows on the same channel, bloated buffers and an unfair 
share of  the aggregate throughput on uplink connections makes its usage in the 
wireless environment far from desirable.

It is evident that Multipath TCP has to be improved to better serve devices in a
wireless environment. We have discussed one of such improvements in 
\S\ref{sec:closing:inflated:cwnd}. Furthermore, to mitigate the effect of 
same-channel interference, Multipath TCP can implement a mechanism where by 
interfaces on the same channel defer transmitting data on that channel 
simultaneously.

In general, Multipath TCP is an exciting technology that has the potential to 
provide more robust network connections between hosts and we believe that with 
further research on its interaction with WiFi, its widespread deployment is 
inevitable.

% Notes from our meeting with Mark about the report:
% - Important: Conclusions - how well does MPTCP works, when does it
%   work/doesn't work, why, and what could be done to improve it (it's all about
%   the conclusions)
% - The main goal of our project is to suggest whether or not MPTCP should be
%   improved, and if so, conceptually what could be done to improve it. We
%   should also mention what would we do next if the project continued.
% vim:textwidth=80:colorcolumn=80:spell:
