In this paper, we have presented a number of wireless experiments showing that
multiplexing across multiple wireless interfaces on a single host may indeed
provide advantages for certain network
configurations both in terms of reliability and throughput.

Furthermore, we have shown that Multipath TCP works well when traffic is
primarily downlink, which is often the case on wireless networks today, as 
traffic is balanced between the available wireless networks and each host
uses a fair share of the total network capacity. We have also shown that 
it behaves unfairly with uplink traffic due to a combination of 
802.11 retransmissions and effectively unbounded buffering in the network stack.

Finally, we have discussed how the uplink problems experienced by Multipath TCP
are not unique to wireless links, and that they manifest also for regular TCP
connections as inflated buffers. Unfortunately, due to time constraints, we have 
not been able to test our suggested solution to this problem, and so the 
question of how to properly limit queue sizes without the presence of packet loss 
remains an open problem.

Unfortunately, there
are many potentially interesting experiments we have not been able to perform due to
the limited time of the project. First and foremost, all of our experiments were 
performed in a building where several other wireless networks were present, 
making it difficult to obtain consistent results. To verify the behaviour we 
have observed, repeating the experiments in more controlled environments may be 
an appropriate next step.

Most of our experiments were also conducted with two static APs, and either two 
or three wireless clients each with a maximum of two interfaces. Investigating 
how Multipath TCP behaves when more networks are available, or with different 
physical distributions of APs, may yield significantly different results. 
As 5~GHz networks become more prevalent, testing the extent and impact of interference 
between 5~GHz channels could also be of value. 

Additionally, we have shown that Multipath TCP experiences varying levels of 
interference based on the underlying network. We see that there is no 
interference between the 2.4 GHz and 5 GHz bands, while the interference patterns 
observed in \S\ref{sec:results-interference} suggest that some further research 
on how Carrier Sense works across channels could be warranted.

Reduced throughput
for flows on the same channel, inflated buffers and unfair behaviour for uplink
traffic makes the deployment of Multipath TCP for wireless clients problematic. To be useful in a
wireless setting, Multipath TCP will need to be modified to correct these issues.
Our proposeal for one such improvement in
\S\ref{sec:results-inflated}
couled be all that is needed, but further research on this topic is needed.

It should be noted that Multipath TCP has been demonstrated to perform
well on wired networks\footnote{C. Raiciu, S. Barre, C. Pluntke, A. Greenhalgh, 
D. Wischik and M. Handley. Improving Datacenter Performance and Robustness with 
Multipath TCP, \textit{ACM Sigcomm 2011}}, and most of the issues identified
in this report are in relation to its interaction with WiFi.

Multipath TCP is an exciting technology, which we believe has the potential to 
provide greater reliability and improved throughput to wireless hosts. With
further research on its interaction with WiFi, we believe it could greatly improve the
experience of using the Internet on mobile devices.

% Notes from our meeting with Mark about the report:
% - Important: Conclusions - how well does MPTCP works, when does it
%   work/doesn't work, why, and what could be done to improve it (it's all about
%   the conclusions)
% - The main goal of our project is to suggest whether or not MPTCP should be
%   improved, and if so, conceptually what could be done to improve it. We
%   should also mention what would we do next if the project continued.
% vim:textwidth=80:colorcolumn=80:spell:
