Before evaluating the performance of Multipath TCP on wireless links, it is
worth exploring whether the nature of wireless permits multiplexing across
multiple wireless links simultaneously. In particular, we want to determine:

\begin{enumerate}
  \item are there any perfomance penalties to broadcasting on two interfaces at
    the same time?
  \item is there a performance win to be had when broadcasting on two
    interfaces at the same time?
  \item to what degree is the downlink performance effected by multiple clients
    using different networks simultaneously?
\end{enumerate}

To answer these questions, we ran several experiments using the different
setups outlined in section TEST-SETUPS, varying the channel frequencies of the % TODO: fix section reference
two networks. One network was generally kept on channel 1 in the 2.4 GHz band,
whilst the other was switched between channel 1 for same-channel tests, channel
5 or 11 for cross-channel tests and a channel in the 5 GHz band. We will state
what network combinations were being used for all results below.

\subsubsection{Interference between 2.4 GHz channels}

Before getting into TCP, we performed some tests using netperf's UDP\_STREAM
benchmark. This simply generates UDP packets in a busy loop, meaning we are
effectively saturating the link. Figure~\ref{graph:cc-interference} shows such a
test with two interfaces on a single host being connected to wireless networks
on different channels. The test was run as outlined in TEST-SETUPS.            % TODO: fix section reference

\begin{figure}[h]
 \centering
 \input{graphs/mp-int-2.4-ni.tex}
 \caption{Cross-channel interference (channels 1 and 11)}\label{graph:cc-interference}
\end{figure}

The interference pattern observed in Figure~\ref{graph:cc-interference} is quite
interesting as the throughput and utilization measurements show a completely
fair sharing between the two networks when both are active. There wireless cards
also report nearly no 802.11 retransmit failures. This perfect split implies
that carrier sense is being employed here, effectively doing time multiplexing
between the interfaces. This would be expected if the two networks were on the
same channel and should not transmit simultaneously, but this experiment was
performed with two networks on opposite ends of the 2.4 GHz spectrum, and the two
interfaces should be able to both transmit at the same time, giving twice the
throughput.

Looking at the sum throughput, it is clear that some performance gain is
achieved, but clearly the aggregate throughput is closer to 150\% of that of a
single interface than the 200\% one would expect from completely independent
channels.

\begin{figure}[h]
 \centering
 \input{graphs/mp-int-2.4-i.tex}
 \caption{Same-channel interference}\label{graph:sc-interference}
\end{figure}

Figure~\ref{graph:sc-interference} is the same test, but performed when the two
networks are on the same channel in the 2.4 GHz band. The results here are
perhaps even more surprising as carrier sense should enforce a fairly strict
time multiplexing with two networks on the same channel, meaning the total
throughput should remain almost the same whether one or two interfaces are
active. The results of our experiments on the other hand show that the
\textbf{total} throughput decreases to almost 60\% of that of a single link when
both interfaces are active.

In both the cases above, it seems as if carrier sense is not doing quite what it
is supposed to. For two networks on different subbands, one would expect carrier
sense to very rarely block transmissions on the other channel, but clearly this
is not the case. For networks on the same subband, carrier sense is expected to
effectively time multiplex fairly between clients, but in our experiments we are
seeing a much higher amount of link layer transmit failures, suggesting that
carrier sense if often failing to prevent collisions.

This leads us to believe that attempting to use two wireless networks at the
same time may in fact lead to a performance penalty compared to using only a
single interface if those two networks are on the same channel. If they are on
different channels, a performance win seems likely, although a lower one than
might be expected.

% TODO: 2.4 down test?

\subsubsection{Interference between bands}

\begin{figure}[h]
 \centering
 \subfloat[][Uplink] {\
   \scalebox{0.55}{\input{graphs/interference-5-2.4-up.tex}}\label{graph:cb-interference-up}
 }
 \subfloat[][Downlink] {\
   \scalebox{0.55}{\input{graphs/interference-5-2.4-down.tex}}\label{graph:cb-interference-down}
 }
 \caption{Cross-band interference}\label{graph:cb-interference}
\end{figure}

Most of the experiments we performed with one network on the 5 GHz band and the
other on 2.4 GHz showed little interference between them as expected.
Figure~\ref{graph:cb-interference-up} shows the CDF of the throughputs for an
uplink test with three of the test described in section TEST SECTION, namely   % TODO: fix section reference
the parallel test and the two single tests. We see that the throughputs are
almost identical in the parallel case and in the single case, implying that the
two bands are indeed independent. That the throughput in the 2.4 GHz parallel
test seems to be marginally lower than for single case is simply a result of the
tests not being run at the same time. Other tests using the same setup show the
2.4 GHz parallel throughput to be higher than that of the single.

Figure~\ref{graph:cb-interference-down} shows a similar test to the above, but
run on the downlink. Again we see the same interference pattern (or rather lack
thereof) with the throughput on one interface being unrelated to whether the
other band is being used or not. This plot also shows a case where the 2.4 GHz
parallel is in fact performing better than its single counterpart. As mentioned
above, we put this down to the background traffic on the network changing
between the times of the two tests.

% TODO: Why is the CDF slope steeper for the up than the down test?
