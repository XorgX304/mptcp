Before evaluating the performance of Multipath TCP on wireless links, it is
beneficial to explore whether the nature of wireless permits transmitting
concurrently across multiple wireless links at all. In particular, we want
to determine:

\begin{enumerate}
  \item Are there any inherent performance penalties to transmitting on two
    interfaces concurrently on the uplink?
  \item Can any performance benefit be gained when transmitting on two
    interfaces concurrently on the uplink?
  \item To what degree is the downlink performance affected by multiple clients
    using different networks simultaneously?
\end{enumerate}

To answer these questions, we ran several experiments using the UDP and TCP New 
Reno protocols, with the different setups outlined in \S\ref{sec:met:setups}.      % TODO: fix section reference
The channel frequencies of the two networks were varied between experiments. One
network was generally kept on channel 1 in the 2.4 GHz band, whilst the other
was switched between channel 1 for same-channel tests, channel 5 or 11 for
non-overlapping channel tests and a channel in the 5 GHz band for the dual band 
test. 

\subsubsection{Interference between 2.4 GHz channels}

\begin{figure}[h]
 \centering
 \input{graphs/mp-int-2.4-ni.tex}
 \caption{Non-overlapping channel interference (channels 1 and 11)}\label{graph:cc-interference}
\end{figure}

Figure~\ref{graph:cc-interference} shows a UDP test as described in 
\S\ref{sec:met:setups:intudp}.
The interference pattern observed here is quite interesting, as the throughput
and utilization measurements show completely fair sharing of aggregate
throughput between the two networks when both interfaces are active. The
wireless cards also report very few 802.11 retransmit failures.

As the two networks are operating at opposite ends of the 2.4 GHz spectrum, we
would expect both interfaces to transmit concurrently without colliding
and the aggregate throughput should be approximately the sum of the capacity of
both links. However, this perfect split of utilisation between the two networks
implies that carrier sense is being used between the interfaces.

Looking at the sum throughput, it is clear that some performance gain is
achieved, but the aggregate throughput is closer to 150\% of that of a single
interface than the 200\% one would expect from completely independent channels.

\begin{figure}[h]
 \centering
 \input{graphs/mp-int-2.4-i.tex}
 \caption{Same-channel interference}\label{graph:sc-interference}
\end{figure}

Figure~\ref{graph:sc-interference} shows a similar test performed with both
networks operating on the same channel in the 2.4 GHz band. The results here are
perhaps even more surprising, as carrier sense should enforce a fairly strict
time multiplexing between two networks on the same channel. We would therefore
expect that the total throughput should remain almost the same, regardless of
whether one or two interfaces are active. However, these results show that the
\textbf{total} throughput decreases to almost 60\% of that of a single link when
both interfaces are active.

In both the cases above, it seems as though carrier sense is not working quite
as expected. We would generally expect few backoffs due to carrier sense between
two networks on non-overlapping channels, but clearly this is not the case.
Conversely, for networks on the same channel we would expect carrier sense to
time-multiplex fairly between the clients. These tests show a high number of 
link layer transmit failures, suggesting that carrier sense if often failing 
to prevent collisions.
% How many? Need to quantify this? J

These results lead us to believe that using multiple wireless networks at the
same time may in fact lead to a performance penalty, as compared to using a
single interface, in the case where the networks are operating on the same 2.4
GHz channel. In the case where the networks are operating on non-overlapping 2.4
GHz channels a performance benefit seems possible, although the gain may be less
than one would reasonably expect.

% TODO: 2.4 down test?

\subsubsection{Interference between bands}

\begin{figure}[h]
 \centering
 \subfloat[][uplink] {\
   \input{graphs/interference-5-2.4-up.tex}\label{graph:cb-interference-up}
 }
 \\
 \subfloat[][downlink] {\
   \input{graphs/interference-5-2.4-down.tex}\label{graph:cb-interference-down}
 }
 \caption{Cross-band interference}
\end{figure}

As expected, the experiments we performed with one network operating on the 2.4
GHz band and one network on the 5 GHz band showed little interference between
the interfaces.  Figure~\ref{graph:cb-interference-up} shows the CDF of the
throughputs for an uplink test with three of the tests described in 
\S\ref{sec:met:setups:seqtcp}, namely the parallel test and the two single tests.
 We see that the throughputs are almost identical in the parallel case and in 
the single case, implying that the networks on these two bands are indeed 
operating independently. The throughput of the 2.4 GHz parallel test is 
marginally lower than in the single test; we attribute this simply to the tests 
not being run at the same time. Other tests using the same setup show the 2.4 
GHz parallel throughput to be higher than that of the single.

% A test at the same time would be nice?

Figure~\ref{graph:cb-interference-down} shows a similar test to the above, but
run on the downlink. We see the same lack of interference between the
interfaces, with the throughput on one interface being unrelated to whether the
other band is transmitting or not. This plot also shows a case where the 2.4 GHz
parallel is in fact performing better than its single counterpart. As mentioned
above, we put this down to the background traffic on other networks changing
between the times of the two tests.

The results indicate that there should be no performance penalty to using
multiple networks on different bands simultaneously. In fact, in the best case,
the total throughput should be the sum of each links' capacity.

% TODO: Why is the CDF slope steeper for the up than the down test?
% vim:textwidth=80:colorcolumn=80:spell:
