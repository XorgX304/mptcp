\subsection{TCP}
\label{sec:bg:tcp}
The Transmission Control Protocol (TCP) is one of the core internet protocols.
It provides reliable, in-order delivery of data between two
systems, and operates in full-duplex. This means that it allows simultaneous
transmission in both directions. It is a packet-based
protocol, and uses positive acknowledgement of packets to signal successful
delivery. TCP also provides flow and congestion
control, which will be explained later in this section.

After establishing a connection between two communicating parties, TCP allows
each host to pass data to it, which is split into packets and sent over the
network to the other host. The Maximum Segment Size (MSS) of a connection limits
how much data TCP is allowed to put into a single packet, and TCP generally
tries to ensure that every packet is exactly one MSS large to avoid sending
more packets than strictly necessary. To avoid overloading the network, or
the receiving host, TCP uses two rate
control mechanisms: flow control and congestion control.

Flow control is used to
avoid sending data at a faster rate than the receiving host is able to process it.
Every packet a host sends out contains an indication of the amount of
buffer space available at the host, and the sender must ensure that it
does not transmit data that the receiver does not have room for. This is
implemented using a sliding window algorithm where each host limits the number
of packets in flight at any given time, making sure to never put more data on
the network than the receiver says it can handle.

Congestion control aims to prevent a TCP flow from sending faster than the
network can support. Upon receiving packets, routers and other devices in a
network must assign some buffer space for these incoming packets to process them
before forwarding along the path, and to deal with any temporary slow-down of the
next link. If a device in the network receives packets faster than it can
retransmit them, the buffer will fill up, and eventually packets are dropped
when the buffer is full. This packet loss indicates congestion, and TCP
attempts to limit itself so that it does not congest any network device along
the path, whilst using an equal, fair share of the available capacity.

The canonical congestion control algorithm is TCP New Reno\footnote{RFC 6582 -
The NewReno Modification to TCP's Fast Recovery Algorithm}. This algorithm
maintains a congestion window, similar to the receiver's flow control window,
which specifies the number of packets which may be in flight concurrently
without overloading the network. The sender must respect both the receive and
congestion windows when deciding whether or not it may send a packet.

New Reno flows begin in a slow start phase. A small initial congestion window
is chosen, and this is increased by one MSS for each packet acknowledgement
received (also known as an ACK). This has the effect of increasing the
congestion window size exponentially until a congestion event occurs. A
congestion event is anything that indicates that a loss has occurred, and is
either a timeout while waiting for an ACK, or the receipt of multiple
duplicate ACKs.

Two mechanisms are used for backing off when congestion is detected. In the
case of a missed ACK, the algorithm returns the congestion window to its small
initial size and begins slow start again. Since the retransmission timeout in TCP
must be
fairly long to cater for high-latency links, a fast recovery mechanism is also
implemented. When three duplicate ACKs are received, TCP considers this as another
indication of congestion, and will halve its congestion window. This is called a
fast retransmit, since it is intended to prevent TCP from having to wait for a
timeout and re-enter slow start. After a fast retransmit, New Reno runs in
additive-increase mode, where the congestion window is increased by one MSS
every round-trip time (RTT). As TCP probes for available capacity and backs off
when it encounters congestion, a distinctive saw-tooth pattern of the congestion
window size is typically produced.

\clearpage
\subsection{WiFi and Interference}
\label{sec:bg:wifi}
The 802.11 standards are a set of physical and Medium Access Control (MAC) layer
specifications for
implementing WiFi networks. 802.11 networks commonly operate on frequencies in
the 2.4 GHz and 5 GHz bands, which are divided into a number of overlapping
sub-bands, commonly known as channels. For example, the 802.11g channels in
the 2.4 GHz band are 22 MHz wide and spaced 5 MHz apart, beginning with channel
1 on 2.412
GHz.

Since many of the 802.11 sub-bands are overlapping, transmissions on one channel
can interfere with signals transmitted on neighbouring channels. Given that the
2.4 GHz and 5 GHz bands are
not specifically reserved for WiFi, they are also susceptible to noise from
other devices such as microwave ovens and Bluetooth-enabled electronics.
Wireless networks are therefore notoriously volatile, and performance can vary
drastically from one location to another, and from one minute to the next.

A typical WiFi deployment consists of several ``stations'' connected to
the network via an access point (AP). A station in 802.11 is typically a single
wireless network interface card (or NIC), and a single machine can have multiple
such NICs, and would thus appear as multiple stations to the AP.\@ The greater
the number of stations connected to an AP, the higher the probability that the
stations would interfere with each other as they contend for the access to the
medium. Multiple stations transmitting simultaneously
on interfering channels is undesirable, as it leads to corrupted packets,
causing retransmissions, thus lowering network utilization and
consequently throughput.

In order to prevent this, 802.11 implements carrier sense and random back-off. Before
transmitting, a station will first listen to determine if another transmission is already
in progress. If the medium is busy, the station will defer for a
random period of time and retry. This behaviour is also used when multiple
stations begin transmitting at the same time, and collide. The random back-off
selected by each station reduces the probability of the stations interfering
with each other on the next retry. Ideally, stations should not carrier sense,
and thus defer to, stations transmitting on non-overlapping channels as this would prevent stations
that could transmit simultaneously from doing so.

Carrier sense will eventually allow a single station to transmit while the
others are waiting, but which station this will be is effectively random. All
stations that have data to send have an equal chance of being allowed to use the
medium, and this is what leads to 802.11 fairly distributing access to the
wireless medium per station.

The 802.11 MAC layer also implements its own packet acknowledgements in addition
to those provided by TCP.\@ Acknowledgement packets are sent immediately after
each data frame is successfully received, in a reserved time slot where no station
may transmit. A station will generally retry transmitting a fixed number of times
without receiving acknowledgement before dropping a packet.

\clearpage
\subsection{Multipath TCP}
\label{sec:bg:mptcp}
Devices with multiple network interfaces are becoming increasingly common. Many consumer
smart phones have both WiFi and 3G interfaces, and data centre networks are
often deployed such that equipment racks are connected through multiple paths.
Data centres themselves are often multihomed to improve reliability, meaning
that they have several points of access to the wider internet. Multipath
TCP\footnote{RFC 6824 - TCP Extensions for Multipath Operation with Multiple
Addresses.} is an extension to TCP recently standardised by the IETF, which aims
to improve redundancy and throughput by taking advantage of multiple paths for a
single TCP flow. In order to achieve this, Multipath TCP adds additional
``subflows'' to a TCP connection so that data flows between every pair of
network interfaces of each host, taking advantage of potentially independent
paths.

% Reference RFC 6182
% ** I think these are actually specified in 6356! **

The authors of Multipath TCP give three main design goals when discussing the
Coupled congestion control algorithm used with Multipath TCP\footnote{RFC 6356 - Coupled Congestion Control for
Multipath Transport Protocols.}:

\begin{description}
  \item[improve throughput] A Multipath TCP flow should perform at least as well
    as a single-path TCP connection would perform.
  \item[do no harm to other network users] A Multipath TCP flow should not take
    up more capacity on any one path than if it was a single path flow using
    only that route -- this is particularly relevant for shared bottlenecks.
  \item[balance network resources] A Multipath TCP flow should balance congestion by moving
    traffic away from the most congested paths.
\end{description}

Coupled thus attempts to behave fairly when links are shared by other TCP or
Multipath TCP flows. In particular, a set of Coupled sub-flows should not gain
more throughput than a competing TCP flow. However, the Coupled algorithm also
aims to utilise the available
links fully when they would otherwise be idle, meaning no other competing flows
are active. This is the case where Coupled can lead to improved throughput over
regular TCP.

The Coupled algorithm maintains a separate congestion window for each subflow
and uses the same congestion avoidance mechanisms as TCP New Reno, but links the
additive increase across all subflows to ensure fairness. For each packet
acknowledgement received on subflow $i$, the congestion window is increased as
shown in Equation~\ref{eq:mptcp}.

% Reference RFC here
\begin{align}
  cwnd_i &= cwnd_i +
    \min\left(\frac{\alpha}{cwnd_\text{total}}, \frac{1}{cwnd_i}\right)\label{eq:mptcp} \\
  \intertext{where}
  \alpha &=
    \frac{cwnd_\text{total} \cdot \max_i\left(\frac{cwnd_i}{rtt_i^2}\right)}
         {(\sum_i \frac{cwnd_i}{rtt_i})^2}\label{eq:alpha}
\end{align}

The parameter $\alpha$ specified in Equation~\ref{eq:alpha} controls how
aggressive Coupled should be when increasing its total send rate, and is chosen
such that the aggregate throughput across all subflows is equal to the
throughput a TCP New Reno flow would gain on the best of the paths available.
This ensures fairness with competing TCP flows, when there is a shared
bottleneck in the network.  Additionally, the algorithm forces the congestion
window of more congested links to increase at a slower rate than less congested
links. This has the effect of shifting traffic onto less congested paths, which
helps to balance congestion in the network.

If the available links are idle, the Coupled algorithm will allow each subflow to
use the full capacity available to it. For example, with two idle links the aggregate
throughput of a Multipath TCP flow should be the capacity of both links combined.

Understanding the intuition behind \textit{why} Coupled works can be difficult
solely using the equation above. We will therefore try to give a less formal
 explanation of what Coupled is doing. As the throughput of a Coupled
flow approaches the throughput a New Reno flow would get on the best link
available to it, Coupled gradually decreases the aggressiveness ($\alpha$)
of all subflows. Doing so effectively decreases the growth rate of each flow's
congestion window. The congestion window of each subflow will continue to increase,
gradually filling the pipe, but it will do so more slowly than New Reno. This means that
should any other flow appear with a growth rate closer to that of New Reno, its
window will grow faster than the Coupled flow's window. When a congestion event
occurs, the window size of both the competing flow and the Coupled
flow will be halved, but since the competing flow is increasing it's window size faster,
it will gain some of the available capacity previously held by the Coupled flow.
This process will continue until Coupled detects that it is at risk of
performing worse than regular TCP, at which point it will go back to increasing its
congestion window as aggressively as New Reno.\@ When this happens, the flows will
reach an equilibrium where they both increase their congestion windows equally
fast, and thus end up sharing the link evenly.

\clearpage
\subsection{Motivation}
\label{sec:bg:motivation}
Multipath TCP is undoubtedly a useful extension to regular TCP, but it was
primarily designed for wired networks in which links are independent.
The nature of wireless means that interfaces can interfere with each
other, which is something that Multipath TCP was not designed to handle.
If Multipath TCP
tries to use multiple interfaces at the same time, the self-interference may
prove sufficiently strong that the benefits of Multipath TCP are effectively
negated.

In this paper, we aim to investigate the extent to which wireless networks
interfere with each other, and how Multipath TCP behaves when used with
non-independent interfaces. To explore this, we analyse its behaviour in a
series of wireless experiments. Our results are presented in
\S\ref{sec:results}.
% vim:textwidth=80:colorcolumn=80:spell:
