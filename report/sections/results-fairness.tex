In the previous section, we showed that Multipath TCP can benefit from using
multiple wireless links simultaneously. As expected, we also saw that Multipath
TCP with New Reno was using an unfair share of the networks' resources;
approximately half of the total available bandwidth when there was a competing
flow on each network. One of the goals of Coupled congestion control is
bottleneck fairness; more specifically, a Multipath TCP flow should not take
more of the bottleneck capacity than any other flow. The following section
evaluates the extent to which Multipath TCP with Coupled congestion control
achieves fairness on wireless networks.

As in the previous section, we will evaluate Multipath TCP's behaviour on the
downlink first, and then on the uplink.

\subsubsection{Downlink fairness}

\begin{figure}[h]
 \centering
 \input{graphs/down-fair.tex}
 \caption{Traffic distribution, downlink, non-overlapping 2.4 GHz}\label{graph:down-fair}
\end{figure}

On the downlink, the Coupled algorithm acts like regular TCP New Reno on clients
with only a single interface; since they only use one subflow each, Coupled will
not try to limit how aggressively it sends data. For the client with multiple
interfaces, the Coupled algorithm should limit the aggregate throughput to as
explained in \S\ref{sec:bg:mptcp}.\@ Figure~\ref{graph:down-fair} shows that
Coupled is indeed limiting one of its subflows and transmitting most of its
traffic on the other. Additionally, its total throughput is close enough to that
of the parallel flows that we can consider Coupled to be behaving fairly; the
client with multiple interfaces is correctly taking only a third of the total
network capacity, as opposed to half in the case of New Reno.

% Don't show send queue?

In \S\ref{sec:results-mptcp-down} we saw that congestion feedback in the
downlink case is provided to the sender by the AP dropping packets if they
arrive at a faster rate than they can be sent out out on the wireless channel.
The Coupled algorithm uses this feedback to correctly reduce its aggressiveness
and thus avoid using an unfair share of the overall capacity. Similar results
are seen when the two wireless links are operating on separate bands, as seen in
Figure~\ref{graph:cb-fairness-down} where Coupled is again limiting itself to
the throughput of the best available link.

\begin{figure}[h]
 \centering
 \input{graphs/cb-fairness-down.tex}
 \caption{Cross-band downlink test with Coupled}\label{graph:cb-fairness-down}
\end{figure}

\subsubsection{Uplink fairness}
\label{sec:results-fairness:up}

\begin{figure}[h]
 \centering
 \input{graphs/up-fair.tex}
 \caption{Traffic distribution, uplink, 5 and 2.4 GHz}\label{graph:up-fair}
\end{figure}

Initially, we expected Multipath TCP with Coupled congestion control to perform
similarly with both uplink and downlink traffic. However, we soon found that
this was not the case. Most of the time plots and CDF graphs showed that each
subflow was performing as well as parallel on \textbf{both} links as seen in
Figure~\ref{graph:up-fair}, indicating that the client with multiple interfaces
was not behaving fairly. Instead, its subflows were taking more of the capacity
in total than any of the single-flow clients as seen with TCP New Reno in
\S\ref{sec:results-mptcp}. After seeing this result in same-channel,
non-overlapping and cross-band tests, we reasoned that the problem had to be
related somehow to the behaviour of WiFi with uplink traffic in general. To
investigate this, we looked for common elements in the time plots of several
tests where Coupled was behaving unfairly.

\begin{figure}[h]
 \centering
 \input{graphs/fairness-up-close.tex}
 \caption{Congestion window size for Multipath TCP uplink connection}\label{graph:fairness-up-close}
\end{figure}

Figure~\ref{graph:fairness-up-close} shows the congestion window as a function
of time for one of the subflows in one such test. In particular, the bandwidth
delay product shows the approximate number of packets we expect the interface to
have in flight at any given point in time. This was calculated by taking the RTT
reported by \texttt{ping} and multiplying it by the average throughput we were
seeing on that interface. The measured RTT was taken when the network was idle,
so it was doubled for this calculation. This gives $20\text{ms} \cdot
10\text{Mbits} \approx 25\text{kB}$. The congestion window of the link is not
expected to surpass this for any significant period of time, yet the graph
clearly shows the congestion window is consistently much greater than this size.

Since Coupled uses the congestion window size to determine how aggressive it
should be, an inflated congestion window will make it more aggressive than the
environment would suggest, resulting in exactly the behaviour we see in
Figure~\ref{graph:up-fair}.

To understand what is happening here, some background knowledge about how TCP
works in Linux, and how the Linux kernel manages network queues is needed. Linux
maintains the IP send queue as a list of pointers to data in the socket buffer.
This list is allowed to grow very large, and so it will never overflow in
practice. When TCP tries to send a packet, it will first check that there is
room in the current receive and congestion windows, and only then will it pass
the packet to the send queue. The NIC then takes packets from the send queue in
FIFO order to transmit them onto the medium.

Ignoring the advertised receive window, the only two limiting factors to how
quickly a TCP flow can send data are therefore the congestion window and the
link speed of the NIC.\@ If TCP passes packets into the queue faster than the
NIC can transmit them, the send queue will simply grow, but not drop packets.
TCP will consider the packet as sent, but the packet will be kept in the IP send
queue until the NIC is ready to transmit it. The relevant effect of this is that
TCP will include the time a packet is queued locally in its estimate of the RTT.

This behaviour is not evident on all wireless networks, however. A TCP flow will
usually start experiencing loss when it overflows a buffer at a bottleneck, and
this is usually not the immediate link as WiFi is generally faster than a user's
Internet speed. The flow will eventually limit the number of packets it sends
based on the capacity of the bottleneck, and since the capacity of the immediate
link is higher than that of the bottleneck, the NIC should be able to send at
least as fast as TCP passes packets into the queue.

A mismatch of assumptions occurs when the WiFi link \textbf{is} the bottleneck
of the path, such as for local connections or when the user has a very fast
connection to the Internet. With the WiFi link as the bottleneck, no loss is
expected to occur elsewhere on the path. The NIC will take a packet from the
queue, use carrier sense and potentially retry some number of times before
actually sending the packet. It will, however, very rarely actually drop a
packet because it usually succeeds in sending the packet before reaching its
retry limit. Since TCP does not see any loss, it will assume that the path is
not congested, and so it will continue to grow its congestion window. In fact,
the loss that TCP does see will likely \textbf{not} be related to congestion,
but rather to faulty links or interfaces, which would cause it to back off its
congestion window incorrectly.

With an ever-increasing congestion window, TCP will continue to put data in the
send queue faster than the NIC can send it causing the queue to become bloated.
This becomes an issue because any packets sent by other flows on the same host
will now be queued behind these, increasing their delay significantly and
negatively impacting applications such as VoIP, where latency is more important
than throughput.

To mitigate this problem, TCP Small Queues has been implemented in the Linux
kernel, effectively hard capping the number of bytes a TCP flow can keep in the
IP send queue at any given point in time. Although this does not solve the
underlying problem, it will limit the amount of delay a misbehaving TCP flow can
inflict on other flows.

For Multipath TCP with Coupled congestion control, however, this becomes a
substantial problem. As explained in \S\ref{sec:bg:mptcp}, Coupled tries to not
be too aggressive on any one link by throttling the growth of the congestion
window. This works correctly if the congestion window directly impacts how often
the NIC puts packets on the network, as throttling the growth of the congestion
window will effectively lead to other clients on the network getting a larger
share of the network's capacity. However, when the congestion window is
sufficiently large that the NIC always has packets to send, this throttling has
no effect at all. The NIC is always able to send, regardless of how quickly the
congestion window grows beyond that point, which means that each
\textbf{subflow} is only limited by the MAC layer fairness, which is fair
\textbf{per interface}. Thus, Coupled will end up behaving exactly like New
Reno, which is unfair when multiple subflows are used.

This problem is not related to the amount of interference between the different
wireless networks, but rather to the lack of feedback given on wireless networks
when the link capacity is reached. It is also only experienced when the WiFi
link is the bottleneck, as any bottleneck further along the network path is
likely to have a queue which will overflow, drop a packet, and thus signal to
TCP that it should back off. In fact, this is precisely what we see in the
downlink case where the queue in the AP effectively conveys how congested the
wireless link is to the sender.

\begin{figure}[h]
 \centering
 \subfloat[][CDF] {\
   \input{graphs/fairness-rtt-up-cdf.tex}\label{graph:fairness-rtt-up-cdf}
 }
 \\
 \subfloat[][Multipath TCP time plot] {\
   \input{graphs/fairness-rtt-up-close.tex}\label{graph:fairness-rtt-up-close}
   % TODO: congestion window is not /10 here, throughput is *10
   % update oddities section calculation to match this
 }
 \caption{Uplink, non-overlapping 2.4 GHz, +100ms}\label{graph:fairness-rtt-up}
\end{figure}

\subsubsection{Uplink fairness on high latency links}
\label{sec:fairness:latency}

Figure~\ref{graph:fairness-rtt-up} shows a test in which the RTT was
artificially increased by 100ms, by adding delay at the IP queue level on the
test server. In this case, we would not expect the higher RTT to significantly
affect overall throughput, only to lower the rate of growth of the congestion
window during slow-start. The CDF, however, shows that throughput is drastically
reduced for all flows.

In particular, consider the time plot of one of the Multipath TCP flows in
Figure~\ref{graph:fairness-rtt-up-close}. The bandwidth delay product shows the
product of the real RTT + 100ms and the throughput observed without an inflated
RTT.\@ Clearly, the congestion window is never approaching this value. The same
is the case for the other subflow and the two parallel flows.

By examining the number of transmission failures reported by the NIC, we can see
that the number of dropped packets is similar to that observed in tests without
an artificially high RTT.\@ This kind of constant loss causes both Coupled and
New Reno to continuously halve their congestion windows. With a higher RTT, the
additive increase is now too slow to bring the congestion window up to the
bandwidth delay product before the next packet loss, after which the congestion
window is halved again. Since TCP is now putting packets into the send queue at
a much lower rate than the NIC can transmit them, the throughput is limited by
the size of the congestion window. Since the window is too small, the throughput
suffers.

Coupled also suffers noticeably more from this problem than New Reno. This is
because Coupled is attempting to behave fairly to other flows, but is reacting
to misleading signals. Coupled reduces its aggressiveness if it detects that
there are competing flows, and this is again deduced from the loss rate. Since
the loss rate is close to constant, Coupled believes that there are other
competing flows, and thus reduces its aggressiveness. This further limits the
growth of each subflow's congestion window, which exaggerates the problem of the
congestion window being too small, further reducing throughput.

% TODO: We may need a better explanation for the above paragraph.

% TODO: mention that this is the case for certain non-wireless networks as well?
% TODO: should we discuss possible solutions here or somewhere else?
% vim:textwidth=80:colorcolumn=80:spell:
