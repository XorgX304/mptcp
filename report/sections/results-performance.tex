\S\ref{sec:results-interference} showed us that there is a performance
gain when using multiple wireless networks if the two networks are somewhat 
separated in the frequency domain. From \S\ref{sec:results-mptcp}, we know that 
Multipath TCP has the potential to fully saturate both of these links if given 
the chance. Our uplink experiments in \S\ref{sec:results-fairness} suggest that 
Coupled will, if it thinks it is not inconveniencing any other flow, gladly fill 
its entire MAC layer share of the network capacity. This section will attempt to 
determine whether Coupled will eventually fill both its wireless links if there 
are no other clients on the network at the same time.

For the uplink case, we already know that Coupled will eventually use the
capacity of the link. Because of the lack of congestion feedback discussed in
\S\ref{sec:results-fairness}, the congestion window will grow too large,
meaning there will always be packets waiting in the send queue, and thus the
only limiting factor is MAC layer fairness. If no other clients are competing
for access, the MAC will give 100\% capacity to the Coupled flow, and thus it
will saturate the link. Until this lack of feedback is somehow resolved, the 
results from experiments with Coupled on the uplink are not particularly
interesting.
%TODO (Is uninteresting the right word?)
% Changed it ^

When Coupled is running for a downlink connection, however, congestion feedback
\textbf{is} given, and so its behaviour when there are no competing flows is 
interesting. To address this, we performed a series of 
experiments where a parallel test was run and then immediately followed by a 
test where one host was connected to both hosts and running Coupled. Due to the 
timing difference between the parallel and Coupled tests, getting identical 
throughput is unlikely, but we would be surprised to see a consistent difference 
in throughput between parallel and Coupled at this point.

\begin{figure}[h]
 \centering
 \input{graphs/coupled-performance.tex}
 \caption{Coupled downlink performance, idle links}\label{graph:coupled-performance}
\end{figure}

The graphs in Figure~\ref{graph:coupled-performance} show the downlink results
for the aforementioned experiments, and we can see that Coupled is undoubtedly
following the performance profile of the independent interfaces. This implies
that Coupled is in fact allowing Multipath TCP to eventually get a total
throughput equal to the sum of what TCP New Reno would get on each flow
individually when there are no competing flows.
% vim:textwidth=80:colorcolumn=80:spell:
