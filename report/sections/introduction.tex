The Transport Control Protocol has been one the Internet's core technologies
almost since its inception, providing reliable delivery of packets across
unreliable networks. However, TCP was only designed to support flows going from
a single interface on one host to a single interface on another host, meaning
that it cannot effectively make use of the increasing connectivity of modern
devices.

Multipath TCP is a recently standardised technology that attempts to solve this
problem. It extends TCP so that a single TCP connection can be distributed
across any number of distinct physical links, allowing it to both provide load
balancing, fail-over, and in some cases, increased throughput. It has been
tested extensively on wired networks and on hosts with WiFi and 3G interfaces,
but is mostly untested on devices connected to multiple WiFi networks. WiFi
networks present an additional challenge as links may interfere with each other,
and it is not immediately clear that putting traffic on one network will not
severely impact other networks.

In this report, we evaluate the behaviour of Multipath TCP when using multiple
WiFi interfaces on a single host; specifically with regard to how
self-interference affects its ability to provide reliability and improved
throughput. We also discuss how fair the protocol is to other clients on the
WiFi networks it uses, as this is a major design goal of Multipath TCP's Coupled
congestion control.

We first give some background information about TCP, WiFi networks and Multipath
TCP in \S\ref{sec:bg}, followed by an explanation of how we performed our
experiments in \S\ref{sec:met}. The main body of the report is in
\S\ref{sec:results}, where we present our results and discuss their implications
for Multipath TCP.\@ Finally, \S\ref{sec:conclusion} discusses our findings as a
whole, as well as pointing out interesting areas for future work.
% vim:textwidth=80:colorcolumn=80:spell:
