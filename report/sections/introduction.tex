Multipath TCP is a recent extension to TCP which allows a single connection to make use of multiple subflows, each taking a different path through the network. Modern smartphones commonly  have both WiFi and 3G connectivity, and data centres often provide multiple paths between top of rack network devices or to the wider internet. By using multiple subflows, a Multipath TCP connection provides greater reliability and throughput than a regular TCP flow.

Interference between wireless devices is a common problem. The 802.11 WiFi standard implements a carrier-sense mechanism to ensure fair and exclusive access to the medium. It is also possible that a device with multiple wireless interfaces may cause self-interference, either by transmitting simultaneously on overlapping frequencies or due to a lack of RF shielding in the device itself.

In general, Multipath TCP has been designed for wired networks. This raises questions about how Multipath TCP would perform in a wireless setting, where a device connected to multiple networks may suffer from self-interference. This report is an investigation into the degree of self-interference caused by Multipath TCP over WiFi networks, and the effect of this on throughput and fairness.

The remaining sections present a brief background of Multipath TCP and WiFi, and then discuss the experiments conducted, results and evaluation.