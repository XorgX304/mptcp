With the extension of Multipath TCP to regular TCP it is possible to have
multiple simultaneous connections on one device. On modern mobile devices that
usually have more than one network interface(wifi, 3G) this can gaurantee extra
reliablity making the connections more robust to loss. With the promise of
greater throughput and reliability, it is immediately easy to see usage
scenarios that involve  multiple wifi cards on a laptop in order to benifit from
MPTCP.

Interference is a bane of any wireless network and wifi is no exception to this.
Multiple devices with single wifi card often interfer with each other and rely
on 802.11 carrier-sense to ensure fair access to the wireless medium. Hence it
is only natural that multiple wifi cards on a device would lead to some degree
of interference.

The goal of this report is to investigate the degree of interference(if any)
between multiple wifi interfaces on the same device, how this interference
influences throughput and fairness to other devices on the network, and how this
influences the multipath TCP protocol. The remaining part of this report would
present a brief background on TCP, Wifi, and Multipath TCP, the methodology of
the experiments conducted, results and evaluation of these experiments and the
conclusions reached.
