Interference between wireless devices is a common problem. The 802.11 WiFi
standard implements a carrier sense mechanism to ensure fair and exclusive
access to the medium. Carrier sense works well when links are in agreement on 
how to share the medium, with far apart links transmitting concurrently and 
nearby links time-multiplexing. This is generally the case for two separate 
devices each with a single wireless interface. However, it is also possible that 
a device with multiple wireless interfaces may cause self-interference, either 
by transmitting simultaneously on overlapping frequencies or due to a lack of RF
(Radio Frequency) shielding in the device itself.

In general, Multipath TCP has been designed for wired networks. This raises
questions about how Multipath TCP would perform in a wireless setting, where a
device connected to multiple networks may suffer from self-interference. This
report is an investigation into the degree of self-interference caused by
Multipath TCP over WiFi networks, and the effect of this on throughput,
fairness and reliability.

As the use of wireless devices become more prevalent in society, Multipath 
TCP needs to surpass the hurdles presented by these networks. The findings 
presented in this report contribute a vast amount into the behaviour of Multipath TCP on Wifi. This information is necessary for the further development and 
deployment of a more robust Multipath TCP protocol on WiFi.

The remaining sections present a brief background of Multipath TCP and WiFi, and
then discuss the experiments conducted, results and evaluation.
% vim:textwidth=80:colorcolumn=80:spell:
