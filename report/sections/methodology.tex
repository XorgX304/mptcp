In order to establish how well Multipath TCP works in its default configuration
with multiple WiFi interfaces, we performed a host of wireless experiments that
measure Multipath TCP's behaviour in different wireless environments and with
different configurations.

We first outline the different testing setups used for the experiments in
\S\ref{sec:met:setups} and the equipment and software we used in
\S\ref{sec:met:equip}. \S\ref{sec:met:metrics} then describes the
data that was recorded during the each test. Finally,
\S\ref{sec:met:scripts} gives an overview of the different scripts used
for running, analysing and plotting tests.

\subsection{Experiment setups}
\label{sec:met:setups}
All experiments we performed used one of the following configurations, usually
for both cross-channel 2.4 GHz, same-channel 2.4 GHz and 5 Ghz configurations.

\subsubsection{Physical layout}
For the most part of the tests the wireless clients would be placed next to each
other with about equal spacing. The APs were placed on a table two meters away
at the same height as the clients, spaced approximately 30cm apart. For the
tests in \S\ref{sec:met:setups:simtcp}, the clients were about a meter
further away due to the need for more space for the extra client.

\subsubsection{Sequential TCP tests}
\label{sec:met:setups:seqtcp}
This test setup consisted of running five tests one after another, and then
plotting all the results in a single graph. For each test, a TCP connection was
initiated from each client to a test server, and data was pushed through the
connection as fast as the congestion control mechanism would allow it. Two APs
were used; one hosting network A and one hosting network B. The five tests were
as follows:

\begin{enumerate}
  \item Machine A connected to network A
  \item Machine B connected to network B
  \item Machine A connected to network A and machine B connected to network B
  \item Machine A connected to network A and network B
  \item Machine B connected to network A and network B
\end{enumerate}

The two first tests were primarily used as baselines for the other tests to see
how well each network performed on its own without interference from the other
network.

The parallel test is used to measure how well two simulateneous, but physically
different, wireless clients would perform on each network. This was done in
order to measure how well Multipath TCP utilized the two networks compared to
two entirely separate, vanilla TCP clients.

Running the dual-network configuration on both machines may seem unnecessary,
but it is vital to be able to correctly compare the parallel performance on a
given channel to the Multipath TCP performance on the same channel. The reason
for this is that wireless behaves slightly different on different machines and
in different locations. Running the dual-network test on both machines allows us
to compare the use of network A for single, parallel and Multipath TCP without
worrying about machine discrepancies. Note that a side-effect of this is that
the Multipath TCP lines in a plot are \textbf{not} from the same test, and thus
would not be expected to match up entirely. That said, we usually observe close
to the exact same behaviour for Multipath TCP for the two last tests.

Because of this discrepancy, most of the results presented in this report are
from the simultaneous TCP test setup described in the next section.

\subsubsection{Simultaneous TCP tests}
\label{sec:met:setups:simtcp}
This test setup had two APs running two different networks and three wireless
clients. The middle client had two wireless interfaces with one connected to
each network. The two other clients had only a single interface each, and were
connected to either of the two networks.

The test itself was a simple TCP streaming test as used in the sequential TCP
tests, but running on all the three clients simultaneously. This setup allowed
us to more directly compare the performance of a Multipath TCP client to that of
two simultaneous clients. It also let us evaluate the fairness of Multipath TCP
when other clients are using the network at the same time.

This testing setup also avoids the cross-host comparison issue exhibited by the
sequential tests, and consequently we mainly include results from the
simultaneous tests in this report.

\subsubsection{UDP interrupt tests}
\label{sec:met:setups:intudp}
UDP interrupt tests are run with only a single wireless client with two wireless
interfaces. Each wireless interface was connected to one of the APs as in the
other setups, but initially data is only transmitted on one of the interfaces.
The other interface alternates between being in idle and active mode. The
alternating interface is idle or active for fixed time slots during the
experiment.

The idea behind alternating the active periods for one of the interfaces is
simple; ideally, if there is no interference between these interfaces, then
there would be close to no loss of throughput for the always-on interface when
the alternating interface is active. If there is interference (such as with
networks on the same 2.4 GHz channel), the throughput of the always-on interface
should drop significantly when the other interface is active.

To avoid having the throughput be limited by other factors than the wireless
channels, interrupt tests are run with UDP rather than TCP.\@ This ensures that
the only limiting factor for the througput is the NIC itself.

\subsection{Equipment and software}
\label{sec:met:equip}
The WiFi interfaces used in our experiments were mostly 2.4 GHz Wi-Pi dongles
from Element 14 commonly found in Raspberry Pi devices. For the 5 GHz tests, a
% TODO: confirm model
Tenda W322U
was used. These dongles did prove somewhat flaky and would occasionally drop
substantial amounts of packets or fail altogether, particularly in uplink tests.
For some experiments we were therefore forced to use the built-in wireless
interfaces on the laptops to perform the experiments. These exhibited
more stable loss rates and generally did not fail during experiments, but also
made comparing results across tests more difficult.

Our APs were two 2.4 GHz
% TODO: confirm model
NETGEAR ProSafe
as well as a
% TODO: confirm model
D-Link
which was used exclusively for 5 GHz tests. All tests were performed with WPA2
encryption enabled and using 802.11g only.

% TODO: confirm models
For wireless clients we used personal laptops including a Lenovo ThinkPad X1
Carbon, a Dell, an Alienware, a MacBook and a Fujitsu-Siemens.

The test servers were two dedicated stationary machines connected to the UCL
internal network using a gigabit switch. Since the APs were all 100 megabit
only, there should be no bottleneck in the network beyond the APs.

To run the tests, NetPerf was used to generate TCP or UDP traffic; the netem
module for qdisc let us emulate network delays and the \texttt{ss} tool provided
invaluable information about internal TCP socket information.

At the time of these experiments, both the servers and laptops were running the
most recent version of the Multipath TCP kernel, version 0.87, which is based
on Linux 3.10.

\subsection{Metrics}
\label{sec:met:metrics}
To understand our results, it is important to know where the data comes from.
During each test, the script \texttt{mp-stats} was run every 500ms and it is the
primary source of data for every plot. The most interesting metrics are
discussed below:

\begin{description}
  \item[throughput]
    Throughput is calculated by looking at the difference in the number of bytes
    sent on each interface (as reported by \texttt{/proc\-/net/\-dev}) and
    dividing it by the time since the last run of \texttt{mp-stats}.
  \item[congestion window]
    Retrieved from the command \texttt{ss -inot} which shows TCP socket
    information. The congestion window size is reported in number of packets,
    and so we multiply it by the mss to get the real congestion window size.
  \item[roundtrip time]
    TCP's RTT estimate is also extracted from the output of \texttt{ss}.
  \item[send queue size]
    Printed by \texttt{ss}, but does not show the size of the IP send queue as
    suspected. It also includes the size of any packets sent, but not yet ACKed
    by TCP.\@ This phenomenon is further examined in
    \S\ref{sec:results-closing-remarks}.
  \item[TCP retransmissions]
    This information is provided in \texttt{/proc\-/net/\-netstat}, but not per
    connection or per interface, only on a global level.
\end{description}

In order to make multiple metrics easier to plot on a single graph, many of the
graphs shown in this section are drawn using our \texttt{mp-plot} script, which
scales values to keep them in the range $[0,100]$. This scaling is important to
understand in order to interpret the results correctly. For example, throughput
is measured in Mbps; utilization is a measure of what percentage of total
throughput is sent through each interface and is displayed so that 80 is
0\% and 100 is 100\%; and congestion window and send queue are both plotted in
10s of kilobytes. The CDF graphs were drawn using the \texttt{mp-cdf} script
which is detailed in \S\ref{sec:met:scripts}.

\subsection{Scripts}
\label{sec:met:scripts}
In order to automate oft-performed tasks such as running tests and analyzing
data, we developed several scripts that were then used throughout the
experiments. The most interesting ones are outlined below.

Note that many of these scripts perform magic based on what wireless networks
the host machine is connected to. For this paper, two servers were used: fry and
zoidberg. Three wireless networks were set up, named bender-wifi, fry-wifi and
zoidberg-wifi. Every test involved at least one of fry-wifi and zoidberg-wifi,
and some of the scripts below use the presence of a connection to one of them as
an indicator of which server should be used for tests.

\begin{description}
  \item[mp-start and mp-congestion]
    These two scripts enable MPTCP on the current machine, as well as set the
    appropriate congestion control algorithm on both the local machine and any
    remote machines it might be connected to.\ \texttt{mp-start} also stops any
    other wireless connections as well as some common network-intensive
    applications such as Dropbox. This prevents other traffic interfering with
    the tests.
  \item[mp-routes]
    Examines the ip addresses of any active network interfaces and sets up
    routing tables according to
    \href{http://multipath-tcp.org/pmwiki.php/Users/ConfigureRouting}{http://multipath-tcp.org/\-pmwiki.php/\-Users/\-ConfigureRouting}.
  \item[mp-run]
    The primary testing script for MPTCP experiments. First, logs information
    about test location, connected networks, nearby wireless networks, TCP
    configuration parameters and kernel version to name a few. Then it starts
    the \texttt{mp-stats} logging daemons to record state information during the
    experiments. It then starts the test itself by running netperf for a
    configurable period of time. When the netperf test is done, the logging
    daemon is stopped and any large log files are compressed.

    \texttt{mp-run} also supports doing downlink tests by spawning a local
    netperf server and running the netperf client on one of the server machines.
  \item[mp-stats]
    Collects the majority of statistics during a test. By default it samples
    data once every second. It logs stats from the wireless interfaces (signal
    strength, bitrate, retransmit failures), IP (bytes and packets sent) and TCP
    (queue sizes, rtt estimates, retransmits)
  \item[mp-int]
    Works much the same as mp-run, but instead of running TCP sessions with
    netperf on all connected interfaces, it runs a UDP\_STREAM test continuously
    on one interface and periodically on the other connected interface. This
    script effectively runs the UDP interrupt tests described in
    \S\ref{sec:met:setups:intudp}.
  \item[mp-analyze]
    Given a test directory created by \texttt{mp-run} or \texttt{mp-int},
    \texttt{mp-analyze} will extract information from various log files and
    output a simple space-separated file for each interface (and a total) with
    values for everything from throughput to bitrate. This information is then
    used by \texttt{mp-plot} or \texttt{mp-cdf} to display graphs or other
    statistical information about the data.
  \item[mp-plot]
    Given a test folder, \texttt{mp-plot} will simply graph every statistic
    generated by \texttt{mp-analyze} for every interface the given test was run
    with. It also performs scaling to keep all values in a 0-100 range as
    discussed in \S\ref{sec:met:metrics}.
  \item[mp-cdf]
    Given tuples of test folders and APs, will calculate the CDF for each
    corresponding interface in each test and graph them using gnuplot. The
    script uses the statistical programming language R to generate the CDF (or
    technically, the ECDF).
  \item[mp-set]
    This script is a shortcut to avoid having to type repeated folder/AP
    names to plot certain data sets. It tries to find all interfaces across
    tests connected to the same channel, and then plot each group of such
    interfaces using \texttt{mp-cdf}.
  \item[mp-merge]
    Merges the data from several test sets into a single set. Optionally also
    does a simple form of normalization in order to make the results more
    relevant when the merged set of samples are plotted as a single CDF.\@ The
    normalization is performed by finding the average of the median throughput
    in each test in the set, and then subtracting that from every throughput
    measurement. This retains both the shape and width of the CDFs, while
    ignoring the absolute throughput values which can vary quite a lot from one
    test to another due to other clients using the network.
  \item[mp-gather]
    Simple wrapper around \texttt{mp-merge} that takes folders of test sets as
    arguments, extracts what WiFi networks were used and calls \texttt{mp-merge}
    for all related tests. For example, it will find all same-channel, coupled
    test sets in all its arguments, and merge them, optionally using the
    normalization feature.
\end{description}
