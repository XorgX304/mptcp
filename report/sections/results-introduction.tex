In the following sections, we will present the results of our experiments and
evaluate how Multipath TCP performs on a host with multiple wireless
interfaces.

\S\ref{sec:results-interference} examines the effect of wireless
interference on simultaneous clients irrespective of Multipath TCP.\@ This is
done to determine to what extent this will affect the quality of the individual
wireless links.

\S\ref{sec:results-mptcp} presents the results of running a Multipath TCP
connection over wireless with New Reno congestion control. The purpose of this
is to investigate whether a Multipath TCP client with both wireless networks to
itself can ever reach the same throughput as two TCP flows running on two hosts
on the different networks.

\S\ref{sec:results-fairness} evaluates the extent to which Multipath TCP behaves
fairly when competing with other flows on the wireless networks. For the uplink
case we also give an analysis of how and why Multipath TCP exhibits unfair
behaviour.

\S\ref{sec:results-performance} examines the performance of Coupled Multipath
TCP on idle paths and tries to determine whether Coupled will ever let a
multipath flow fully utilize idle wireless links.

In \S\ref{sec:results-reacting} we present results showing how Multipath TCP
reacts to changes in a mobile environment, and finally
\S\ref{sec:results-closing-remarks} gives some closing thoughts and explains
several strange phenomenon we observed during our wireless experiments.
% vim:textwidth=80:colorcolumn=80:spell:
