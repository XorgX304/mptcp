In the following sections, we will present the results of our experiments and
evaluate how Multipath TCP performs on a host with multiple wireless
interfaces.

\S\ref{sec:results-interference} examines the effect of wireless
interference on simultaneous clients irrespective of Multipath TCP.\@
To determine to what extent interference affects the quality of the
individual wireless links.

\S\ref{sec:results-mptcp} presents the results of running a Multipath TCP
connection over WiFi with New Reno congestion control, to investigate whether
a Multipath TCP client with both wireless networks to
itself can ever be expected to reach the same throughput as two regular TCP flows running on
separate hosts on the different networks.

\S\ref{sec:results-fairness} evaluates the extent to which Multipath TCP behaves
fairly when competing with other flows on the wireless networks. For the uplink
case we also give an analysis of how and why Multipath TCP exhibits unfair
behaviour in some situations. \S\ref{sec:results-inflated} then discusses a
potential solution to this problem.

\S\ref{sec:results-performance} examines the performance of Coupled Multipath
TCP on idle paths, and tries to determine whether Coupled will ever allow a
multipath flow fully utilize idle wireless links.

In \S\ref{sec:results-reacting}, we present results showing how Multipath TCP
reacts to changes in a mobile environment. Finally, in
\S\ref{sec:results-closing-remarks} we give some closing thoughts and explain
several strange phenomenon we observed during our wireless experiments.
% vim:textwidth=80:colorcolumn=80:spell:
