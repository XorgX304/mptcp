In the following sections, we will present the results of our experiments and
evaluate how Multipath TCP performs on a host with multiple wireless
interfaces available to it.

Section~\ref{sec:results-interference} examines the effect of wireless
interference on simultaneous clients irrespective of Multipath TCP.\@ This is
done to determine to what extent this will affect the quality of the individual
wireless links.

Section~\ref{sec:results-mptcp} presents the results of running a Multipath TCP
connection over wireless without Coupled congestion control. The purpose of this
is to show whether an MPTCP client with both wireless networks to itself can
ever reach the same throughput as two TCP flows running on two hosts on
the different networks.

Section~\ref{sec:results-fairness} evalues to what extent Multipath TCP behaves
fairly when competing with other flows on the wireless networks. For the uplink
case where Multipath TCP proves to violate the second rule of Multipath TCP; do
no harm, we give an analysis of why Multipath TCP exhibits unfair behaviour in
this case.

Finally, section~\ref{sec:results-oddities} explains several strange phenomenon we
observed during our wireless experiments.
