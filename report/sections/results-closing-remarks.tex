Throughout our experiments, we have come across several odd results,
correlations and trends which we could not immediately explain. Some of these
turned out to be glitches, but many made complete sense after some hard
thinking. This section aims to explain most of the oddities that can be observed
in some of the graphs given in this report.

\subsubsection{Reliability of wireless experiments}
Running each test many times has proven very important throughout this project.
Considering the uplink New Reno 2.4 GHz tests as an example, we initially 
suspected that Multipath TCP was consistently seeing slightly lower 
throughput compared to two separate flows in parallel, and we surmised that
this might be because two 
WiFi interfaces connected to one machine might somehow lead to more 
cross-interface interference than with the same interfaces connected to 
different machines equally far apart. After running more experiments to confirm this, 
however, we quickly saw contradicting results where Multipath TCP was 
consistently faster than the parallel case. These kind of discrepancies were
present in many of our tests, and are due to WiFi being extremely sensitive to
positioning and timing. We therefore had to repeat many experiments to obtain
consistent results. 

\subsubsection{Inflated RTT}
A number of our tests experienced very high RTTs of around 600 ms despite the network
being relatively fast (i.e.\ we see an RTT of $\approx 7$ ms with \texttt{ping}).
The RTT also seemed to increase and decrease with the size of the send queue. As 
discussed in \S\ref{sec:results-fairness}, the lack of packet loss causes 
unbounded growth of the congestion window, but with most of the packets actually being
the local send queue. Since TCP estimates RTT based on when the packet was put into 
the send queue, \textbf{not} when it is actually sent by the network interface, 
the RTT estimate will include the time a packet spent in the queue. Since the 
queue is growing, so will the RTT.

\subsubsection{Logarithmic growth of congestion window}
Figure~\ref{graph:logarithmic} shows the congestion window size plotted over
time for one test we ran in which we were seeing very little loss. It clearly
shows something resembling logarithmic growth rather than the familiar linear
saw-tooth. To understand why this is happening, it is necessary to look closer
at how the congestion window is increased.

The congestion control mechanisms used by Multipath TCP and regular TCP
increase the congestion window by one MSS per RTT.\@ They do this by increasing
the congestion window by approximately $\sfrac{1}{cwnd}$ per received ACK.\@
This works well in the expected case where increasing the congestion window will
cause more packets to be sent, and thus receive more ACKs per RTT, but is not
correct when the link layer masks loss.

What happens when TCP does not see loss, as discussed in \S\ref{sec:results-fairness}, 
is that the congestion window usually grows larger than what the network 
interface can handle, and so increasing it will not cause any more packets to be 
sent. The number of ACKs received in an RTT will therefore remain constant. The 
$\sfrac{1}{cwnd}$ term, on the other hand, will grow smaller, and so the growth of the 
congestion window relative to the congestion window size per RTT will decrease, 
leading to the logarithmic growth seen in the results.

\subsubsection{Congestion window and the send queue}
\label{sec:closing:sendq}
Figure~\ref{graph:logarithmic}, and some other plots in this report, show the
send queue size tracing the congestion window size through the entire test.
Since we were seeing very little loss in many of our tests, and thus the
congestion window was clearly growing larger than it should be, we initially
reasoned that most of the bytes of the congestion window were likely to be in
the host's send queue. In this case, we would expect the send queue to closely
follow the congestion window, with a small gap between them representing the
packets actually in flight.

However, we then noticed that we were seeing the send queue following the
congestion window also when we \textbf{were} seeing loss and the congestion
window was \textbf{not} inflated. In these cases, there should not be many
packets in the send queue at all; they should mostly all be in flight.

It turned out that the reason for this is surprisingly simple; the send queue
size reported by the kernel includes \textit{unacked packets}. TCP keeps packets
after sending in case they must be re-transmitted, so until they have been
ACKed, they will continue to take up space in the queue. This explains both why
we were not seeing a gap between the send queue size and the congestion window
in the no-loss experiments, \textbf{and} why the send queue was seemingly
following the congestion window when loss was occurring.

\subsubsection{Congestion window and throughput}

In Figure~\ref{graph:fairness-rtt-up-close}, we can observe another very strange
phenomenon that occurred in a number of experiments. Here, we see a curious
correlation between throughput and the congestion window size. Other tests also
show a correlation between the two, but it is particularly evident here as it 
seems like the throughput is bounded by the congestion window.

To determine why this was happening, we first tried to find similarities
between the graphs that were showing this peculiar trend. It turns out that we
were only seeing this in tests which showed either a high RTT with constant loss
rates, or a high amount of loss. These cases both share the feature that the
congestion window is constrained from growing to the full bandwidth delay
product of the path, and so no queues are expected to build up anywhere in the
network. The limiting factor for the throughput is thus the congestion window not
allowing TCP to add more packets to the send queue, even though the link is
ready to send. The net effect of this is that the throughput is limited by the
congestion window; whenever the congestion window grows, the throughput
increases because TCP is allowed to put more packets on the wire. If the
congestion window is halved, TCP stops sending packets almost immediately, and
the throughput drops.
% vim:textwidth=80:colorcolumn=80:spell:
