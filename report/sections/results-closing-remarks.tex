Throughout our experiments, we have come across several rather odd results,
correlations and trends that we could not immediately explain. Some of these
turned out to be glitches, but many made complete sense after some hard
thinking. This section aims to explain most of the oddities that can be observed
in some of the graphs given above.

\subsubsection{Wireless experiments}
Running every test many times has proven very important throughout this project.
For the uplink reno 2.4 GHz tests for example, we initially suspected that
Multipath TCP was consistently seeing slightly lower throughput compared to
parallel, and we surmised that this might be because two WiFi interfaces
connected to one machine might somehow lead to more cross-interface interference
than with the same interfaces connected to different machines equally far apart.
After running more experiments to confirm, however, we quickly got contradicting
results where Multipath TCP was consistently faster than parallel. On balance,
we consider the results to pretty much even out across many tests. The slight
discrepancies of some links in those figures are simply due to WiFi interfaces
being extremely sensitive to their exact positioning, and average out over a
larger set of tests.

\subsubsection{Congestion window and the send queue}
Most of the time plots above show the send queue size tracing the congestion
window size through the entire test. Since we were seeing very  little loss in
many of our tests, and thus the congestion window was clearly growing larger
than it should be, we initially reasoned that most of the bytes of the
congestion window were likely to be in the host's send queue. In this case, we
would expect the send queue to closely follow the congestion window, with a
small gap between them representing the packets actually in flight.

However, we then noticed that we were seeing the send queue following the
congestion window also when we \textbf{were} seeing loss and the congestion
window was \textbf{not} inflated. In these cases, there should not be many
packets in the send queue at all; they should mostly all be in flight.

Our first guess was that the send queue size included the TCP socket buffer
size, but this was quickly discarded as the TCP socket buffer should usually be
close to full (i.e.\ a constant value) due to the netperf tool adding packets in
a tight loop. That was clearly not what was happening here.

It turns out that the reason for this is surprisingly simple; the send queue
size reported by the kernel includes \textit{unacked packets}. TCP keeps packets
after sending in case they must be re-transmitted, so until they have been
ACKed, they will continue to take up space in the queue. This explains both why
we were not seeing a gap between the send queue size and the congestion window
in the no-loss experiments, \textbf{and} why the send queue was seemingly
following the congestion window when loss was occurring.

\subsubsection{Inflated RTT}
Many of our results show very high RTTs of around 600 ms despite the network
being relatively fast (i.e.\ we see an RTT of $\approx 7$ ms with
\texttt{ping}). 

% TODO: set correct section 
The RTT seems to increase and decrease with the size of the send queue. As 
observed in \S\ref{sec:results-fairness}, the lack of loss reference causes 
unbounded growth of the congestion window, but with most of the packets stuck in 
the send queue. Since TCP estimates RTT based on when the packet was put into 
the send queue, \textbf{not} when it is actually sent by the network interface, 
the RTT estimate will include the time a packet spent in the queue. Since the 
queue is growing, so will the RTT.

\subsubsection{Logarithmic growth of congestion window}
When looking at the growth of the congestion window, it is clear that it shows
something resembling logarithmic growth rather than the familiar linear
saw-tooth. To understand why this is happening, it is necessary to look closer
at how the congestion window is increased.

The congestion control mechanisms used by Multipath TCP and regular TCP try to 
increase the congestion window by one MSS per RTT.\@ It does this by increasing 
thecongestion window by approximately $\sfrac{1}{cwnd}$ per received ACK.\@ This 
works well in the assumed case where increasing the congestion window will cause 
you to send more packets, and thus receive more ACKs per RTT, but is not correct 
when the link layer masks loss.

What happens when TCP does not see loss, as discussed in 
\S\ref{sec:results-fairness}, is %TODO: set correction section reference 
that the congestion window usually grows larger than what the network interface 
can handle, and so increasing it will not cause any more packets to be sent. 
The number of ACKs received in an RTT will therefore remain constant. The 
$\sfrac{1}{cwnd}$ term will grow smaller, and so the growth of the congestion 
window relative to the congestion window size per RTT will decrease, leading to 
the logartichmic growth seen in the results.

\subsubsection{Congestion window and throughput}

In Figure~\ref{graph:fairness-rtt-up-close}, we can observe another very strange
phenomenon that occured in a number of experiments. Here, we see a curious
correlation between throughput and the congestion window size. Other tests also
show a correlation between then two, but it is particularly evident in this one
as the throughput looks like it is actually bounded by the congestion window.
This struck us as very odd since the congestion window is plotted in 10s of
kilobytes, whereas the throughput is plotted in Mbit/s.

To determine why this was happening, we first tried to find commonalities
between the graphs that were showing this peculiar trend. It turns out that we
were only seeing this in tests which showed either a high RTT with constant loss
rates, or a high amount of loss. These cases both share the feature that the
congestion window is constrained from growing to the full bandwidth delay
product of the path, and thus no queues are expected to build up anywhere in the
network. The limiting factor for the throughput is the congestion window not
allowing TCP to add more packets to the send queue, even though the link is
ready to send. The net effect of this is that the throughput is limited by the
congestion window; whenever the congestion window grows, the throughput
increases because TCP is allowed to put more packets on the wire. If the
congestion window is halved, TCP stops sending packets almost immediately, and
the throughput drops.

This still does not explain why we see close to a 1:1 correspondence between the 
congestion window and the throughput in the 100 ms RTT test. After
investigation, we determined that this happens because of a curious case of two
different scales accidentally lining up. Remember that the congestion window is
plotted as 10s of kilobytes, and thus we plot $c(x)$ as

\begin{align*}
  c(x) = cw_x \text{\ (B)}
       = \frac{cw_x}{1024} \text{\ (kB)}
       = \frac{cw_x}{10 \cdot 1024} \text{\ (10 kB)}
\end{align*}

now, consider the throughput $t'(cw)$ allowed by a congestion window of size
$cw$ over a link with RTT $\approx 100ms$:

\begin{align*}
  t'(cw) &= \frac{cw}{0.100}                       &&\text{(Bps)} \\
         &= \frac{cw}{0.100 \cdot 10 \cdot 1024}
          = \frac{cw}{1024}                        &&\text{(MBps)} \\
         &= \frac{cw}{8 \cdot 1024}                &&\text{(Mbps)}
\end{align*}

Since the congestion window is not allowed to grow to the bandwidth delay
product as explained above, we expect the real throughput $t(x)$ to be
approximately limited by $t'(cw_x)$. It is clear from the above that $t'(cw) =
0.8 c(x)$, and so in this particular setup, we would expect to see the
throughput follow the congestion window closely at these particular scales
because $t(x) \approx 0.8 c(x)$.
