\documentclass[12pt,a4paper]{article}
\usepackage[vmargin=3cm]{geometry}
\usepackage{setspace}
\usepackage{hyperref}
\title{\vspace{-5ex}Executive Summary}
\author{\vspace{-5ex}}
\date{\vspace{-5ex}}
\begin{document}
\maketitle
\doublespacing
The ever-growing demand for faster and more reliable access to the Internet on a
wide variety of devices has become a moving target in today's increasingly
connected world. To quench this desire for constant connectivity, hardware
manufacturers are continuously adding new communication technologies to their
devices, with many modern laptops having Ethernet, WiFi, Bluetooth, 3G and
occasionally even Near-Field Communication technology embedded in them.

Since the early days of the modern Internet, the Transport Control Protocol, or
TCP, has been one of its core technologies, providing reliable delivery of
packets across unreliable networks. However, TCP was only designed to support
flows going from a single interface on one host to a single interface on another
host, and thus it cannot effectively utilize these additional links. Although
one could build applications that open multiple TCP connections to another host
and multiplexes data across these itself, this would be a major undertaking, and
would likely encounter be difficult to get right.

Multipath TCP is a recently standardised technology that attempts to solve this
problem. It extends TCP so that a single TCP connection can be distributed
across any number of distinct physical links, allowing it to both provide load
balancing, fail-over, and in some cases, an increase in throughput. It allows,
for example, an application to continue working seamlessly as a device leaves a
WiFi network and goes back to a 3G-only connection. It has been tested
extensively on wired networks and on hosts with WiFi and 3G interfaces, and in
these deployments it has been shown to work very well.

With the proliferation of WiFi networks, however, a new source of connectivity
is becoming relevant. If a device has more than one WiFi interface, it could
potentially connect to multiple WiFi networks at the same time, distributing
traffic and transitioning between them as necessary. Unfortunately, WiFi
networks \textbf{do} interfere with each other, and it is not immediately clear
that putting traffic on one network will not severely impact other networks.
Using Multipath TCP for such a configuration might therefore be
counter-productive, as it might cause so much self-interference that any
potential benefits are negated.

The purpose of this report is to evaluate the behaviour of Multipath TCP when
using multiple WiFi interfaces on a single host; specifically with regard to
how self-interference affects its ability to provide reliability and improved
throughput. We will also explore other differences between wired and wireless
networks, in particular how wireless networks may cause Multipath TCP to not
behave fairly to competing flows, even with Coupled congestion control.

Our results show that Multipath TCP can improve both reliability and throughput
when using multiple WiFi links for both downlink and uplink traffic. In both
cases, we show that on two idle wireless networks, Multipath TCP can achieve an
aggregate throughput equal to the sum of both links. We also demonstrate how
Multipath TCP behaves unfairly on the uplink when competing flows are present on
any of the wireless networks.

Experiments with uplink traffic show that the Coupled congestion control
algorithm effectively behaves like regular TCP's New Reno due to a lack of
congestion feedback from the wireless interfaces. We discuss why this might be
the case, revealing that the underlying issue also affects normal TCP traffic,
and present a possible solution that applies to both regular TCP and Multipath
TCP.

Our results suggest that further investigation into Multipath TCP's behaviour on
WiFi is warranted, and that until Coupled or 802.11 is modified, Multipath TCP
may behave very unfairly in scenarios where uplink traffic dominates. We also
suggest that the behaviour of TCP over WiFi links should be more thoroughly
examined, as current implementations may introduce significant additional
latency on end hosts.
\end{document}
% vim:textwidth=80:colorcolumn=80:spell:
