\documentclass[12pt,a4paper]{article}
\usepackage{graphicx}
\usepackage{lmodern}
\usepackage[vmargin=3.5cm]{geometry}
\usepackage{setspace}
\usepackage{hyperref}
\title{Executive Summary}
\date{\vspace{-8ex}}
\begin{document}
\doublespacing
\maketitle
With the widespread of the Internet and its ease of access, being connected has
become a focal point if not an essential staple in our modern, fast-paced
society. Communication has become heavily dependant on it and its various
protocols; one of the core Internet protocols is the Transport Control Protocol
(TCP), it enables reliable communication between two different hosts and most of
the communication carried out nowadays is only made possible because of it.
However, this protocol only enables single path connections between two TCP
flows, but presently more and more devices come equipped with multiple
interfaces.

An effort to address this issue was the creation of an extension to TCP called
Multipath TCP which enables communication using multiple interfaces
simultaneously. This extension aims to improve throughput and redundancy by
adding subflows to a single TCP flow and taking full advantage of this property.

Multipath TCP was designed for a wired environment, and has been proved to show
improvements when compared to the normal TCP (maybe add a paper reference?);
however, with the widespread of wireless communications which tend to be very
unpredictable and susceptible to interference from other sources, it is not
clear how Multipath TCP behaves as it has not been fully investigated. (Do we
need an intro? If not I think we may only need what is written from the next
paragraph onwards)

The purpose of this report is to evaluate the effect of WiFi interference on
Multipath TCP.\@ The purpose and motivation of Multipath TCP is addressed in the
background section of the report, where concepts relevant to Multipath TCP and
the way it behaves are explained in more detail. Concepts such as TCP and WiFi
and Interference.
 
The problem considered in this report and the conclusions drawn from it have
been based on results from several wireless experiments ran using different
settings and layouts. These settings and layouts played an important role in the
behaviour of Multipath TCP.
 
The results presented show that there can be a performance gain in using
Multipath TCP exclusively with WiFi for both the downlink and uplink. Both cases
show that on two idle paths Multipath TCP can attain an aggregate throughput
which is equal to the sum of both links. However, while the downlink contending
traffic results show that the aggregate throughput is reduced to the capacity of
the best single link, thus making its behaviour fair to contending flows, the
uplink behaves quite differently.  Uplink results show that the aggregate
throughput is generally more than that of the best link and only in some cases
equal to the sum of both links. This behaviour shows that Multipath TCP on the
uplink is unfair to other flows on the medium as it gets a larger share in
total. This contradicts the second rule of its default congestion control
algorithm which states "do no harm", and raises questions regarding how well
Multipath TCP behaves with WiFi.

These results are described in further detail in the Results and Evaluation
section of the report, where several graphs are used to illustrate exactly what
type of behaviour Multipath TCP was displaying for each particular setting.
These graphs were invaluable in helping us understand how Multipath TCP works
and reasoning on what could do to improve it.

A possible solution to the behaviour shown in the uplink results is also
presented and even though it has not been implemented due to time constraints,
it thoroughly explains why Multipath TCP behaved the way it did and what could
be done to improve it in theory.

This report suggests that further investigation into Multipath TCP and WiFi
interference should be carried out as there is in fact a performance gain and
also a lot of potential in the improvement of Multipath TCP.
\end{document}
